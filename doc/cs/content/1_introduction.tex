\chapter{Introduction}

\section{Motivation}
Public key cryptography based on elliptic curves allows smaller key sizes compared to other cryptographic systems.
While still providing equivalent security, the smaller key size leads to huge performance benefits.
\\
Blind Signatures are one of the key components upon which Taler's privacy is built upon.
Our thesis adds support for a modern cryptographic scheme called the Clause Blind Schnorr Signature scheme \cite{cryptoeprint:2019:877}.\\
Additionally to the benefits of ellicptic curve cryptography, adding a second blind signature scheme makes Taler independent of a single cryptographic scheme and thus provides \textit{cipher agility}.


\section{Goals}
\label{sec:goals}
The project definition is as follows \cite{project-definition}:

The students will implement the blind Schnorr signature inside Taler.
Taler is a system for the management of virtual money.
Taler is based on coins that need to be signed by an exchange (for instance a bank).
In the actual version of the system, coins are signed by the exchange using Schaum's bind-signature protocol.
This allows users to have signed coins, without the exchange knowing what it signed.
This step is fundamental for the privacy protection of the system.
\\The students have to insert the Schnorr blind signature algorithm inside the protocol for the creation of coins.
But they also need to change the Taler subsystems where the verification of the signature is done.
\\The actual Taler system allows people to let an exchange sign a coin for which they do not have the private key.
This is a security issue (for misuse of coins on the dark-net for instance).
An optional task for the project is to prevent a user to let an exchange sign a public key when the client does not have access to the corresponding private key.
\\Here is a list of the tasks that the students must do:
\begin{itemize}
    \item Design a protocol integrating Schnorr blind signature in the creation of Taler coins.
    \item Implement the protocol inside the exchange application and the wallet app.
    \item Analyze the different Taler subsystems to find where the blind signature is verified.
    \item Replace verification of the blind signature everywhere it occurs.
    \item Compare both blind signature systems (Schaum's and Schnorr's), from the point of view of security, privacy protection, speed, \dots
    \item Write tests for the written software.
    \item Conduct tests for the written software.
    \item Transfer the new software the Taler developers team
\end{itemize}
Here is a list of optional features:
\begin{itemize}
    \item Design a protocol, such that the exchange can verify that the user knows the private key corresponding to the coin that is to be signed.
    \item Implement that protocol.
\end{itemize}

\section{Scope}
\label{sec:scope}
In scope are all necessary changes on the protocol(s) and components for the following tasks:
\begin{itemize}
    \item Research the current state of Blind Schnorr Signature schemes
    \item Redesign the Taler protocols to support Blind Schnorr signatures
    \item Add support for a Blind Schnorr Signature Scheme in the exchange, merchant, wallet-core, wallet web-extension and optionally on the android mobile wallet
    \item design and implement a protocol where the user proves to the exchange the knowledge of the coin that is to be signed (optional)
\end{itemize}

Out of scope is production readyness of the implementation.
This is because changes in the protocos and code need to be thoroughly vetted to ensure that no weaknesses or security vulnerabilities were introduced.
Such an audit is out of scope for the thesis and is recommended to be performed in the future.
The iOS wallet will not be considered in this work.
\\
It is not unusual that a scope changes when a project develops.
Due to different reasons, the scope needed to be shifted.
Since there are no libraries supporting \gls{CSBS}, the signature scheme has to be implemented and tested before integrating it into Taler.
While this is still reasonable to do in this project, it will affect the scope quite a bit.
The analysis of the optional goal showed, that a good solution that aligns with Taler's goals and properties needs more research and is a whole project by itself.

Scope changes during the project:
\begin{itemize}
    \item \textbf{Added:} Implement the cryptographic routines in GNUnet
    \item \textbf{Removed: } design and implement a protocol where the user proves to the exchange the knowledge of the coin that is to be signed (optional)
    \item \textbf{Adjusted: } Focus is on the implementation of the exchange protocols (Withdraw, Spend, Refresh and cryptographic utilities)
    \item \textbf{Adjusted: } Implementation of the refresh protocol and wallet-core are nice-to-have goals
    \item \textbf{Removed: } The Merchant and the android wallet implementations are out of scope
\end{itemize}