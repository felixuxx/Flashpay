\chapter{Conclusion}
This section provides a summary of this work, presents the results and gives an outlook on future work.

\section{Summary}
In the beginning of the project good knowledge on the current state in research about Blind Schnorr signatures was needed.
Therefore, various papers were read and then the paper "Blind Schnorr Signatures and Signed ElGamal Encryption in the Algebraic Group Model" \cite{cryptoeprint:2019:877} was chosen as basis for the redesign of the Taler protocols.\\
The next step was to analyze the current Taler protocols and understand the required properties including \textit{\gls{abort-idempotency}}.\\
With the gathered knowledge (see chapter \ref{chap:preliminaries}) the Taler protocols were redesigned to support \gls{CSBS} (see chapter \ref{chap:design}).
These redesigned protocols were then further specified (chapter \ref{chap:spec}) and then implemented (chapter \ref{chap:implement}).
The implementation includes the main protocols, key management, cryptographic utilities in Taler and the \gls{CSBS} cryptographic routines.\\
The \gls{CSBS} scheme was analyzed and compared in detail to the RSA Blind Signature scheme (see \ref{chap:disc}).


\section{Results}
The thesis provides several results to add support for Schnorr's blind signature in Taler, including:
\begin{itemize}
    \item Redesigned Taler protocols to support \gls{CSBS}
    \item Implementation of cryptographic routines
    \item Implementation of Taler protocols in Exchange
    \begin{itemize}
        \item Key Management and security module
        \item Cryptographic utilities
        \item Withdraw protocol
        \item Deposit protocol
    \end{itemize}
    \item Comparison and Analysis
    \begin{itemize}
        \item Performance (speed, space, latency \& bandwith)
        \item Security 
        \item Scheme Comparison
    \end{itemize}
    \item Fixing a minor security issue in Taler's current protocols
\end{itemize}
The code is tested, and those tests are integrated in the existing testing framework.
Benchmarks are added for the cryptographic routines and the security module.

\section{Future Work}
Like in any other project, there is always more that could be done.
This section provides an outlook on what can be done in future work.

\begin{itemize}
    \item Implement wallet
    \item Implementing remaining \gls{CSBS} protocols (refresh, tipping protocol, refund etc.)
    \item Implementing merchant
    \item Security audit of CS implementation
    \item Find a solution for withdraw loophole
    \item Evaluating \& implementing \gls{CSBS} on other curves
\end{itemize}

There are some remaining protocols to implement, which were out of scope for this thesis. 
To run \gls{CSBS} in production, these protocols have to be implemented too.
Further, the merchant needs to support \gls{CSBS} too.
The merchant implementation can be done fast, as the merchant only verifies denomination signatures in most cases. \\
Currently, the exchange runs both security modules, the \gls{CSBS} and the RSA security modules.
To reduce unnecessary overhead, this should be changed so that only one security has to be running.
To run \gls{CSBS} in production a security audit from an external company is recommended (as done for other parts in the exchange, see \cite{codeblau:taler-audit}).
A security audit should always be made when implementing big changes like these.\\
As mentioned in the scope section, the optional goal to find and implement a good solution for the withdraw loophole was dropped.
This was due to the scope shift and because the analysis of the problem showed that finding a good solution needs more research and is a whole project in itself (see \ref{sec:scope} for more information).\\
Furthermore, \gls{CSBS} could be implemented on other curves.
For example Curve448 \cite{cryptoeprint:2015:625} could be used, as it provides 224 bits of security, wheras \gls{25519} \cite{bern:curve25519} provides about 128 bits of security.
Curve secp256k1 could further improve \gls{CSBS} performance.
While providing support for Curve448 should not be problematic, a potential implementation for secp256k1 needs further analysis (see \cite{bernlange:safecurves} and \cite{bip:schnorr-bitc} for more information).

\section{Personal Conclusion}
This thesis includes understanding, analyzing, integrating and implementing a recent academic paper \cite{cryptoeprint:2019:877} containing a modern cryptographic scheme.
Furthermore, the implementation is done in Taler, an intuitive and modern solution for a social responsible payment system with high ethical standards.
Although there was a lot of work, we enjoyed working on such a modern and very interesting topic.
Especially the first successful signature verification and the signature scheme performance benchmarks motivated us to push the implementation and integration into Taler forward.\\
We are happy to provide an implementation of a modern scheme and making it available as free software.