%!TEX root = ../thesis.tex

%
% vorher in Konsole folgendes aufrufen: 
%	makeglossaries makeglossaries dokumentation.acn && makeglossaries dokumentation.glo
%

%
% Glossareintraege --> referenz, name, beschreibung
% Aufruf mit \gls{...}
%
% \newglossaryentry{non-repudiation}{name={non-repudiation},plural={non-repudiation},description={After a message is signed, one can not dispute that a message was signed}}
% \newglossaryentry{sender_authenticity}{name={sender authenticity},plural={sender authenticity},description={The origin/sender of a message can not be forged}}
% \newglossaryentry{message_integrity}{name={message integrity},plural={message integrity},description={No unauthorized change to the message can be made, the message is tamperproof}}
\newglossaryentry{hkdf}{
    name        =   {HKDF},
    description =   {The HMAC-based Extract-and-Expand Key Derivation Function is a function that takes potentially weak keying material as input and outputs high entropy keying material. For more information see section \ref{sec:kdf}}
}

\newglossaryentry{25519}{
    name        = {Curve25519}, 
    description = {A popular elliptic curve used in many cryptographic systems based on elliptic curve cryptography. See section \ref{par:curve25519}} 
}

\newglossaryentry{fdh}{
    name        = {FDH},
    description = {A Full-Domain Hash is a hash function with an image size equal to the original gorup. See section \ref{sec:rsa-fdh}}. 
} 

\newglossaryentry{idempotence}{
    name        = {idempotence},
    description = {Idempotence in the context of computer science is a property to ensure that the state of system will not change, no matter how many times the same request was made. See section \ref{abort-idempotency}} 
} 

\newglossaryentry{abort-idempotency}{
    name        = {abort-idempotency},
    description = {Abort-idempotency is a special case of \gls{idempotence}. On every step in a protocol it needs to be ensured that even on an abort, the same request always receives the same response. See section \ref{abort-idempotency}} 
} 

\newglossaryentry{RSABS}{
    name        = {RSA Blind Signatures},
    description = {Chaums Blind Signature Scheme based on RSA. See section \ref{sec:blind-rsa-sign}} 
}

\newglossaryentry{CSBS}{
    name        = {Clause Blind Schnorr Signatures},
    description = {A secure variant of Blind Schnorr Signature Schemes introduced in section \ref{sec:clause-blind-schnorr-sig}}
} 

% \newglossaryentry{25519}{
    % name        = {},
    % description = {} 
% } 