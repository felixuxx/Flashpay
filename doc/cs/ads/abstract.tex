\chapter*{Abstract}
GNU Taler is an intuitive, fast and socially responsible digital payment system implemented as free software.
While preserving the customers privacy, GNU Taler is still compliant to regulations.
\\\\
The goal of this thesis is to improve Taler's performance and provide cipher agility by adding support for Schnorr's blind signatures.
To achieve this goal, the current state in research for Schnorr signatures needs to be analyzed.
After choosing a signature scheme, it has to be integrated into the Taler protocols.
Besides implementing the redesigned protocols in Taler, an implementation of the cryptographic routines is needed.
\\\\
The paper "Blind Schnorr
Signatures and Signed ElGamal Encryption in the Algebraic Group Model" \cite{cryptoeprint:2019:877} from 2019 (updated in 2021) introducing \gls{CSBS} is used as theoretical basis for our improvements.
The paper explains why simple Blind Schnorr Signatures are broken and how the Clause Schnorr Blind Signature scheme is secured against this attack.\\
Compared to the currently used \gls{RSABS}, the new scheme has an additional request, two blinding factors instead of one and many calculations are done twice to prevent attacks.
\\\\
The Taler protocols were redesigned to support the Clause Blind Schnorr Signature scheme, including slight alterations to ensure \textit{abort-idempotency}, and then further specified.
Before starting with the implementation of the redesigned protocols, the cryptographic routines for \gls{CSBS} were implemented as part of the thesis. \\
All of the implemented code is tested and benchmarks are added for the cryptographic routines.
\\\\
Multiple results were achieved during this thesis:
The redesigned protocols Taler protocols with support for \gls{CSBS}, the implementation of the cryptographic routines, the implementation of Talers core protocols and a detailed comparison between \gls{RSABS} and \gls{CSBS}.
Overall, the \gls{CSBS} are significantly faster, require less disk space, and bandwidth and provide \textit{cipher agility} for Taler.

\section*{Acknowledgement}
We would like to kindly thank Christian Grothoff (Bern University of Applied Sciences) for his extensive advice, support and very helpful feedback during our whole thesis.\\
We also kindly thank Jeffrey Burdges (Web 3, Switzerland) for reviewing the proposal containing the redesigned protocols and giving feedback.\\
Further, we kindly thank Jacob Appelbaum (Bern University of Applied Sciences, Eindhoven University of Technology) for further results for the performance measurements of our cryptographic routines and the insightful conversations.
